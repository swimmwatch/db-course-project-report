\section{Введение}

\subsection{Формальные требования}
\begin{enumerate}
    \item Необходимо выбрать предметную область для создания базы данных. Выбранная предметная область должна быть уникальной для всего потока, а не только в рамках учебной группы.
    \item Необходимо описать таблицы и их назначение. Выполнить проектирование логической структуры базы данных. Описать схему базы данных. Все реальные таблицы должны иметь 3 нормальную форму или выше. База данных должна иметь минимум 5 таблиц.
    \item Необходимо разработать два клиентских приложения для доступа к базе данных. Данные приложения должны быть написаны на двух разных языках программирования и иметь разный интерфейс (например, классическое оконное приложение и web-приложение). Выбор языков программирования произволен.
    \item Необходимо организовать различные роли пользователей и права доступа к данным. Далее, необходимо реализовать возможность создания архивных копий и восстановления данных из клиентского приложения.
    \item При разработке базы данных следует организовать логику обработки данных не на стороне клиента, а, например, на стороне сервера, базы данных, клиентские приложения служат только для представления данных и тривиальной обработки данных.
    \item Ваша база данных должна иметь представления, триггеры и хранимые процедуры, причем все эти объекты должны быть осмысленны, а их использование оправдано.
    \item При показе вашего проекта необходимо уметь демонстрировать таблицы, представления, триггеры и хранимые процедуры базы данных, внешние ключи, ограничения целостности и др. В клиентских приложениях уметь демонстрировать подключение к базе данных, основные режимы работы с данными (просмотр, редактирование, обновление \dots)
    \item Необходимо реализовать корректную обработку различного рода ошибок, которые могут возникать при работе с базой данных.
\end{enumerate}

\subsection{Предметная область}
Область применения данного приложения универсальна. Можно использовать тестирование на сотрудниках, школьниках, студентах и так далее.

\subsection{Стэк технологий}
\begin{itemize}
    \item \textcite{wiki:js} --- мультипарадигменный язык программирования. Поддерживает объектно-ориентированный, императивный и функциональный стили. Является реализацией стандарта \textcite{wiki:es}.
    \item \textcite{react} --- \textcite{wiki:js} библиотека для создания пользовательских интерфейсов.
    \item \textcite{redux} --- контейнер состояния для \textcite{wiki:js} приложения.
    \item \textcite{react-router} --- набор навигационных компонентов.
    \item \textcite{ts} --- типизированный расширенный набор \textcite{wiki:js}.
    \item \textcite{scss} --- препроцессор, который расширяет \textcite{wiki:css}.
    \item \textcite{wiki:css} --- формальный язык описания внешнего вида документа, написанного с использованием языка разметки.
    \item \textcite{wiki:html} --- гипертекстовый язык разметки.
    \item \textcite{node.js} --- среда выполнения JavaScript, созданная на основе движка Chrome V8 JavaScript.
    \item \textcite{express} --- минимальный и гибкий \textcite{node.js} фреймворк для создания веб-приложений.
    \item \textcite{python} --- язык программирования, который позволяет быстро работать и более эффективно интегрировать системы.
    \item \textcite{postgres} --- объектно-реляционная база данных с открытым исходным кодом.
    \item \textcite{seqorm} --- \textcite{node.js} \acrshort{orm} на основе обещаний для Postgres \textcite{postgres}.
\end{itemize}

\subsection{Инструменты}
\begin{itemize}
    \item \textcite{git} --- система контроля версий.
    \item IDEs: --- интегрированная среда разработки.
    \begin{itemize}
        \item WebStorm
        \item DataGrip
    \end{itemize}
    \item Линтеры --- программы, которые следят за качеством кода.
    \begin{itemize}
        \item \textcite{eslint} --- проверяет качество кода на \textcite{wiki:js}.
        \item \textcite{stylelint} --- проверяет качество кода на \textcite{scss}, \textcite{wiki:css}.
    \end{itemize}
    \item Тестирующие фреймворки:
    \begin{itemize}
        \item \textcite{jest} --- среда тестирования \textcite{wiki:js} с упором на простоту.
        \item \textcite{selenium-python} --- предоставляют простой API для написания тестов с использованием Selenium WebDriver.
    \end{itemize}
    \textcite{webpack} --- сборщик статических модулей для современных \textcite{wiki:js} приложений.
\end{itemize}

\subsection{Организация работы с Git}


\clearpage