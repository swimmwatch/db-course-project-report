\section{Заключение}
Благодаря данному курсовому проекту, я поверхностно освоил разработку \acrshort{spa} приложений с помощью библиотеки \textcite{react} и фреймворка \textcite{express}.

\subsection{Недостатки}
Подводя итоги, мне бы хотелось перечислить вещи, на которые я буду обращать внимание при разработке следующих проектов:
\begin{itemize}
    \item Использование CI/CD.
    \item Использование методологий в вёрстке. Например, \acrfull{bem}. Её разработали внутри компании Яндекс. У них есть свой стек технологий под данную методологию, который облегчает разработку клиентской части.
    \item Использование \acrshort{nosql} баз данных вместе с реалиционными. Хранить \acrshort{json} в таблице плохо, поэтому для этой задачи подходить \textcite{mongodb}.
    \item Разделение клиентского кода на чанки.
    \item Микросервисная архитектура.
\end{itemize}

\clearpage